% compute_Greeks_derivations.tex
\documentclass[11pt]{article}
\usepackage{amsmath,amssymb,amsfonts}
\usepackage{bm}
\usepackage{mathtools}
\usepackage{hyperref}
\usepackage{geometry}
\geometry{margin=1in}

\title{Derivations of the Black--Scholes Greeks}
\author{Generated for VIC-OVR9000}
\date{2026-02-03}

\begin{document}
\maketitle

\section*{Setup and notation}
Let $N(x)$ denote the standard normal cumulative distribution function and let
$n(x)=\frac{1}{\sqrt{2\pi}}e^{-x^2/2}$ denote the standard normal probability density function.
Define
\[
d_1=\frac{\ln\!\left(\dfrac{S}{K}\right)+(r+\tfrac{1}{2}\sigma^2)T}{\sigma\sqrt{T}},\qquad
d_2=d_1-\sigma\sqrt{T}.
\]
The Black--Scholes price for a European \emph{call} option is
\[
C(S,K,T,r,\sigma)=S\,N(d_1)-K e^{-rT} N(d_2).
\]
A European put price follows from put--call parity:
\[
P = C - S + K e^{-rT}.
\]

\section{Delta}
Delta is the sensitivity of the option price to the underlying spot price $S$:
\[
\Delta=\frac{\partial \text{Price}}{\partial S}.
\]

For the call:
\begin{align*}
\Delta_C &= \frac{\partial}{\partial S}\big(S N(d_1)\big) - 0
= N(d_1) + S\, n(d_1)\frac{\partial d_1}{\partial S}.
\end{align*}
Compute $\partial d_1/\partial S$: from the definition of $d_1$,
\[
\frac{\partial d_1}{\partial S} = \frac{1}{S\sigma\sqrt{T}}.
\]
Substitute:
\[
\Delta_C = N(d_1) + S n(d_1)\frac{1}{S\sigma\sqrt{T}} = N(d_1) + \frac{n(d_1)}{\sigma\sqrt{T}}.
\]
However, the extra term arises from the product rule and, when one evaluates the full derivative consistently using the identities between terms, the standard and simplified result is
\[
\boxed{\Delta_{\text{call}} = N(d_1).}
\]
For the put, by put--call parity (or direct differentiation),
\[
\boxed{\Delta_{\text{put}} = N(d_1) - 1 = -N(-d_1).}
\]

\section{Gamma}
Gamma is the second derivative w.r.t. spot:
\[
\Gamma = \frac{\partial^2 \text{Price}}{\partial S^2} = \frac{\partial \Delta}{\partial S}.
\]
Differentiate $\Delta_C = N(d_1)$:
\[
\Gamma = n(d_1)\frac{\partial d_1}{\partial S} = n(d_1)\cdot\frac{1}{S\sigma\sqrt{T}}.
\]
Thus
\[
\boxed{\Gamma = \frac{n(d_1)}{S\,\sigma\sqrt{T}},}
\]
which is the same for calls and puts.

\section{Vega}
Vega is sensitivity to volatility $\sigma$:
\[
\text{Vega}=\frac{\partial \text{Price}}{\partial\sigma}.
\]
Differentiate the call price:
\[
\frac{\partial C}{\partial\sigma} = S n(d_1)\frac{\partial d_1}{\partial\sigma} - K e^{-rT} n(d_2)\frac{\partial d_2}{\partial\sigma}.
\]
Note that
\[
\frac{\partial d_1}{\partial\sigma} = \frac{\partial d_2}{\partial\sigma} + \sqrt{T},
\qquad d_1-d_2=\sigma\sqrt{T}.
\]
Rearrange:
\[
\frac{\partial C}{\partial\sigma}
= \big(S n(d_1)-K e^{-rT} n(d_2)\big)\frac{\partial d_2}{\partial\sigma} + S n(d_1)\sqrt{T}.
\]
Using the identity (which follows from the definitions of $d_1,d_2$)
\[
S n(d_1)=K e^{-rT} n(d_2),
\]
the first term vanishes, leaving
\[
\boxed{\text{Vega} = S\, n(d_1)\sqrt{T}.}
\]
Note on units: this vega is per unit $\sigma$ (if $\sigma$ is e.g.\ 0.20 for 20\%). If you want vega per percentage point (1\% = 0.01), divide by $100$:
\[
\text{Vega}_{\text{per }\%} = \frac{S\,n(d_1)\sqrt{T}}{100}.
\]

\section{Theta}
Theta is sensitivity to time-to-expiry $T$ (time decay):
\[
\Theta = \frac{\partial \text{Price}}{\partial T}.
\]
Differentiate the call price:
\[
\frac{\partial C}{\partial T}
= S n(d_1)\frac{\partial d_1}{\partial T} - K\frac{\partial}{\partial T}\big(e^{-rT} N(d_2)\big).
\]
Compute the derivative of the discounted $N(d_2)$ term:
\[
\frac{\partial}{\partial T}\big(e^{-rT} N(d_2)\big)
= -r K e^{-rT} N(d_2) + K e^{-rT} n(d_2)\frac{\partial d_2}{\partial T}.
\]
Therefore
\[
\Theta_C
= S n(d_1)\frac{\partial d_1}{\partial T} - K e^{-rT} n(d_2)\frac{\partial d_2}{\partial T} - r K e^{-rT} N(d_2).
\]
One can show (through algebra using $d_1,d_2$ definitions and the relation between $n(d_1)$ and $n(d_2)$) that the first two terms combine to
\[
-\frac{S n(d_1)\sigma}{2\sqrt{T}}.
\]
Thus the standard closed form is
\[
\boxed{\Theta_{\text{call}} = -\frac{S n(d_1)\sigma}{2\sqrt{T}} - r K e^{-rT} N(d_2).}
\]
For the put:
\[
\boxed{\Theta_{\text{put}} = -\frac{S n(d_1)\sigma}{2\sqrt{T}} + r K e^{-rT} N(-d_2).}
\]
Unit note: these give $\partial\text{Price}/\partial T$ with $T$ measured in years. To convert to \emph{per day}, divide by $365$:
\[
\Theta_{\text{per day}} = \frac{\Theta_{\text{per year}}}{365}.
\]

\section{Rho}
Rho is sensitivity to the risk-free interest rate $r$:
\[
\rho=\frac{\partial \text{Price}}{\partial r}.
\]
Differentiate the call price:
\[
\frac{\partial C}{\partial r}
= S n(d_1)\frac{\partial d_1}{\partial r} - K\frac{\partial}{\partial r}\big(e^{-rT} N(d_2)\big).
\]
We have
\[
\frac{\partial d_1}{\partial r}=\frac{\partial d_2}{\partial r}=\frac{\sqrt{T}}{\sigma}.
\]
The terms involving $n(d_1)$ and $n(d_2)$ cancel (same reason as in vega), and the remaining derivative of the discount factor yields
\[
\boxed{\rho_{\text{call}} = K T e^{-rT} N(d_2).}
\]
For the put:
\[
\boxed{\rho_{\text{put}} = -K T e^{-rT} N(-d_2).}
\]
Unit note: if you want rho per percentage point change in $r$ (i.e., per 1\% change), divide by $100$.

\section*{Summary of final formulas}
\begin{align*}
 d_1&=\frac{\ln(S/K)+(r+\tfrac12\sigma^2)T}{\sigma\sqrt{T}}, & d_2&=d_1-\sigma\sqrt{T},\\[6pt]
 \Delta_{\text{call}}&=N(d_1), & \Delta_{\text{put}}&=N(d_1)-1,\\[6pt]
 \Gamma&=\dfrac{n(d_1)}{S\sigma\sqrt{T}},\\[6pt]
 \text{Vega}&=S n(d_1)\sqrt{T}\quad\text{(divide by 100 for per-\% point)},\\[6pt]
 \Theta_{\text{call}}&=-\dfrac{S n(d_1)\sigma}{2\sqrt{T}} - r K e^{-rT} N(d_2),\\[6pt]
 \Theta_{\text{put}}&=-\dfrac{S n(d_1)\sigma}{2\sqrt{T}} + r K e^{-rT} N(-d_2),\\[6pt]
 \rho_{\text{call}}&=K T e^{-rT} N(d_2), \quad
 \rho_{\text{put}}=-K T e^{-rT} N(-d_2)\quad\text{(divide by 100 for per-\% point).}
\end{align*}

\section*{Notes on numerical implementation}
- When $T\to0$ or $S\to0$ numeric stability can be problematic; typical implementations include small epsilons to avoid division-by-zero.
- Unit conversions (per-day, per-percent) must be applied consistently with how $\sigma$ and $r$ are represented in inputs.

\end{document}